\chapter{\textbf{Final considerations}}
\label{chap:final}

In this paper, we aimed to solve the virtualization design problem in a private cloud. By implementing the virtualization design advisor over OpenNebula, we showed that this is possible. The advisor is responsible for improving the overall performance of running multiple database workloads in each host from the cloud. We optimize the CPU allocation among different VM guests and rely on OpenNebula for automatizing this task along the cloud.

Through experiments, we showed that we are able to identify the workload needs by leveraging the cost model built within the DBMS. We achieved a gain up to $8\%$ in initial tests. It was also possible to see that the mechanism created to dynamically correct estimative errors in the cost model works effectively. Besides propitiating a improvement of about $10\%$ over the default allocation, it also prevents the advisor from making incorrect allocation decisions, thus causing a deterioration in performance. Along with a module created to detect major changes in the workload, the advisor is able to adapt the cost models to fluctuations in the workload.

%In this paper, we considered the problem of automatically configuring multiple virtual machines, which were deployed in a cloud. The aim was to show that it is possible to improve the overall performance of running database workloads by optimizing the resource allocation among them. Through the development of an advisor and further experimentations, it was demonstrated that this objective can be achieved even by reallocating a single resource.

%Our approach to leverage the cost model built within the query optimizer was shown to be effective. The advantage of its use is that the cost model takes into consideration the nature of database workloads. However, this approach can turn the support of new DBMSes much harder. If a DBMS does not offer tools for self calibration, the tuning parameters that describe its resources must be calibrated manually. Thus, their use within the cost model and their variation need to be analyzed. For the same reason, the support of new resources are also hard.

%We have also shown that without the advisor, all resource allocation levels must be set manually and they are static along the VM's life cycle. This means that they will not be adapted to the workload needs. That is why the existence of the virtualization design advisor in a cloud is important.

Regarding future work, there is a lot of improvement to be made in database consolidation. Obvious directions include expanding the virtualization design advisor, in order to support new resource types and DBMSes. Further experiments could show the effects of changing the workload intensity when running the workloads. Furthermore, the cloud infrastructure could be more exploited. For instance, VM migration could be used to transfer VMs to hosts with more available resources.  This feature is already supported by OpenNebula. The problem is how to optimize these transfers, because they generate a high overhead. 

%However, before expanding this work, some considerations should be made. Our approach to leverage the cost model built within the DBMS can turn the support of new DBMSes much harder. If a DBMS does not offer tools for self calibration, the tuning parameters that describe its resources must be calibrated manually. Thus, their use within the cost model and their variation need to be analyzed. For the same reason, the support of new resources are also hard.