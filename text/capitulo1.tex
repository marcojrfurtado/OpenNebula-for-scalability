\chapter{\textbf{Introduction}}

\label{Introduction}

Cloud computing has attracted a lot of attention lately. It was popularized as a business model by Amazon's Elastic Compute Cloud (EC2), which started selling virtual machines (VMs) in 2006. Over time, more cloud providers have appeared in the market, offering their computational resources (CPU, memory, and I/O bandwidth). This definition of cloud, in which the IT infrastructure is deployed through virtual machines, is referred as Infrastructure-as-a-service (IaaS). One of the most appealing benefits of this paradigm, when associated to cloud providers, is the ability to cut costs. Companies may base their IT strategies on cloud-based resources, spending very little or no money managing their own IT infrastructure. They pay for these resources on-demand, in contrast to the traditional resource provisioning model, in which they would need to deal with both under- and over- utilization of their own resources. Moreover, the cloud providers may offer lower prices because they are benefited from the economy of scale.


Of course, the gains cited above are only noticeable if we are talking about public clouds -- clouds made available in a pay-as-you-go manner to the public by an external provider. Although the market has evolved around this type of cloud, organizations might build IaaS clouds using their own infrastructure, known as private clouds. Their aim is not to sell capacity  over the internet,  but to give local users an agile and flexible infrastructure to run service workloads in their administrative domains. These users are offered VMs, which are scheduled in a group of physical machines within their organization, which we call a cluster. This leads to a better utilization of resources, since services with little demand can be packed into the same machine, process known as server consolidation. Other benefits, such as the migration of VMs between hosts and the ability to dynamically change the amount of resources provided to it, enabled by the  technology present in virtual machine monitors (VMMs), make it possible to deal with fluctuations in the workload. These characteristics propitiate an elastic environment, which is good for a private cloud, and vital for the pay-on-demand model used in public clouds.  It's also possible for an organization to mix these two types of clouds, creating a hybrid cloud,  which is useful to supplement a private cloud's infrastructure with external resources from a public one. However, this paper will not get into the details of hybrid clouds. 

A cloud is highly dependable on machine virtualization, essential to achieve its goals. Database management systems (DBMS), like other software systems, are also increasingly being run on virtualized environments for many reasons. Some of them are mentioned in \cite{4498282}, \cite{4401021} and \cite{Soror:2008:AVM:1376616.1376711}, which include the reduction on the cost of ownership, better provisioning and manageability of applications and the ability to migrate it among physical hosts. This paper focus on other motivation, which is the possibility to take a variety of databases that run on dedicated computing resources and move them to a shared resource pool, including the ability to reallocate these resources as needed. It was formalized as a problem in \cite{4401021}, defined as the virtualization design problem (a.k.a. resource consolidation problem ) for relational database workloads. It can be defined as follows: \textit{"Given N database workloads that will run on N database systems inside virtual machines, how should we allocate the available resources to the N virtual machines to get the best overall performance?"}. According to that paper, the virtualization design problem may find a better solution when applied to relational database systems due to three factors. First, relational database workloads consist of SQL queries with constrained and highly specialized resource usage patterns. Second , queries are highly variable in the way they use resources -- one query might heavily need CPU, while another might need I/O bandwidth instead. Thus, they could benefit from the dynamism in resource allocation. Third, database systems already have a way of modelling their own performance, namely the query optimizer.

As mentioned, DBMSes have particularities involving their workloads. Therefore, the application running inside a VM shouldn't be treated as a black box. Instead, the database system cost model should be exploited. As VMMs have parameters to control the share of physical resources, database systems have tuning parameters to manage their own performance. These two sets need to be simultaneously analysed and tuned. In \cite{Soror:2008:AVM:1376616.1376711}, this is a principle of a proposed \textit{virtualization design advisor}. It works by recommending configuration parameters for a group of VMs, each one containing a DBMS. These parameters determine how the shared resources will be allocated to each VM, and consequently to each DBMS. It uses information about anticipated workloads to specify the parameters offline. Furthermore, runtime information collected after the deployment of the recommended configuration can be used to refine this recommendation and to handle fluctuations in the workload. It has not been proposed to run this advisor in a cloud yet, rather it has been implemented and tested in a single physical machine, in which two DBMS instances were deployed.


This paper proposes an implementation of this virtualization design advisor in a cloud environment, in a distributed manner. The advisor should be able to configure all the VMs deployed in a cluster, considering only the resources of the physical host in which it was deployed. It is expected that this advisor provides a better utilization of resources in the cloud, even though the cloud is not perceived by the advisor. Its efficiency would need to be confirmed through tests that would need to be performed after implementing. The rest of this paper is structured as follows. In Section 2, the \textit{virtualization design advisor} is described. Section 3 is used to show how a cloud infrastructure is managed. In section 4, it is proposed an integration of the advisor within the cloud management system. Finally, section 5 presents some final considerations and an idea about future work.

