\chapter{\textbf{Proposed implementation}}

In this paper, it's proposed an implementation of the virtualization design advisor over OpenNebula. The aim is to optimize the distribution of resources inside the private part of a cloud for VMs running database workloads. Since it's not possible, nor makes sense to optimize an external provider's infrastructure, the public part of the cloud is just ignored. We also stick to the PostgreSQL\footnote{http://www.postgresql.org/} as the DBMS used in our solution, although the support for other DBMSs could be extended in future work.

As already described, the virtualization designer advisor has only been modelled and tested against one server. The major issue of this paper is to port it to a cloud, which can contain several hosts. Other issue is that a cloud is an heterogeneous environment. In spite of the homogeneous view of resources provided by the VMM, the hosts may be different themselves. To address these problems, the proposed approach consists in having $N$ instances of our advisor, being $N$ the number of nodes in the private cloud. Therefore, the advisor will not see the cloud as a whole, but only the host it was assigned to work with, the same as it was working wit only one server. 

The first step of this implementation would be the creation of a module responsible for the calibration process. As discussed earlier, this calibration will be used to map query optimizer's cost model, which depends on its parameters $P_{i}$, to  the advisor's cost model, based on resources $R_{i}$. This process is supposed to be executed before any VM deployment, since the initial allocation depends on the latter cost model. Thus, it will be executed whenever a new physical host is added to a cluster. As this paper is limited to one DBMS, the renormalization step will  not be performed after calibration.


After calibration, we expect the OpenNebula's default scheduler to define which host will be assigned to the VM that is to be deployed, as usual. At this step, the only provision that needs to be taken is to set the \textit{RANK} variable properly. In this paper, we intend to deal with only two types of resources: memory and CPU. Therefore, we propose the following heuristic for the \textit{RANK}:
\begin{itemize}
 \item \textit{RANK} $=$ $\alpha *$ \textit{FREECPU} + $ (1 -\alpha) *$ \textit{FREEMEMORY}, $  0 \le \alpha \le 1  $.
\end{itemize}
In this equation, we expect $\alpha$ to be used to prioritize one type of resource over another. Since the scheduler assigns the host before being possible to run the advisor, this heuristic will be used for all kinds of workloads. So it doesn't matter whether they are more or less CPU intensive or memory intensive. We consider this to be a limitation. However, we still expect that this heuristic will be able to  distribute well the VMs among the hosts.

Once the VMs are placed on a machine, the initial configuration step from the virtualization design advisor can be started. In order to deal with their initial configuration, it's proposed the use of the greedy search algorithm, described earlier in this paper. A problem that has already been identified, which affects the greedy algorithm, is the lack of current support for dynamic resource reallocation in OpenNebula. One possible approach would be the use of libvirt\footnote{http://libvirt.org/}, which defines itself as "A toolkit to interact with the virtualization capabilities of recent versions of Linux (and other OSes)". It offers an API that works with all the hypervisors supported by OpenNebula, offering many features, including resource reallocation. Currently, OpenNebula has an libvirt API, which enables this tool to manage the VMs over the core layer. It should be possible to implement a solution by using libvirt under the core layer, to extend the drivers' capabilities.

The online refinement and the dynamic configuration management steps should have direct implementations, following their descriptions. The initial configuration is defined by the greedy algorithm ( the static resource allocation module ), it stops when it find it's the best allocation, which may be the optimal or close to it. Once this task is performed, the online refinement is started. It updates the cost model by observing estimated and actual times of workloads.  It returns an optimized cost model to the first step, where the greedy algorithm is restarted and searches for a new recommendation with the optimized cost model. The optimizer only stops when the newly obtained recommendations don't differ from the original ones (i.e. it stabilizes ). The dynamic configuration management module is also started after the initial configuration was performed. It also uses optimized cost models to monitor changes in the workload. It may restart the workload when major changes are detected.

The performance of this advisor is expected to improve as optimized cost models are obtained, and so less calls to the DBMS's query optimizer are needed. This is due to the fact that these calls have a high computational cost.


